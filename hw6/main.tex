\documentclass[titlepage,14pt]{article}

\usepackage[utf8]{inputenc}
\usepackage{setspace}
\usepackage[margin=1.0in]{geometry}
\usepackage{fancyhdr}
\pagestyle{fancy}
\fancyhf{}
\renewcommand{\headrulewidth}{0pt}
\fancyhead[R]{\thepage}
\usepackage{lipsum}
\usepackage[english]{babel}

\usepackage{etoolbox}
\AtBeginEnvironment{quote}{\singlespacing\small}

\usepackage[notes,backend=biber]{biblatex-chicago}
\addbibresource{sample.bib}

\title{Player Pianos and Their Significance In Understanding Modern Piano Performance}
\author{Brian Liu}
\date{March 22, 2018}


\begin{document}

\maketitle


\large
\begin{doublespace}

% TODO: thesis paragraph outlining two arguments to be made and through what historical frames/methods
Piano or keyboard performance is an art that has existed since before the 1700s. The goal of piano performance is to create a faithful reproduction of a piano composer's work. For years, piano performance was the only means by which a composer's work could be replayed, using mainly the composer's original written score for musical direction.
%the essential means of making up for a lack of original recording. 
However, with the recollection, rediscovery, and recirculation of player pianos and their rolls, the answer to ``What is a faithful interpretation of a musical work?'' becomes much more complicated. Through the lens of material culture studies and the history of the idea of piano performance, we will examine the history and meaning of player pianos, then analyze the differences between the composer's recordings and piano performance practices of today.

% TODO: introduction to first section

% TODO: define player pianos and player piano rolls
In order to understand the significance of the rediscovery of these recordings, one must understand what a player piano and a player piano roll is. For all intents and purposes, most player pianos look and sound the same as a normal piano, but the main difference between a normal piano and a player piano is that a player piano is fully automated and reproduces mechanical key presses based on inscriptions marked on a sheet of paper, which is fed to the player piano. This sheet of paper, typically in the form of a roll, is called a piano player roll, or piano roll.\autocite{playerpianoproject}

% TODO: section on history of piano rolls and player piano and argument about materialism and "taking piano player out of player piano" - what it means to be ``authentic'' making of music - beginnings of alternative methods of creating music
With these definitions in mind, our first object is the keyboard-less Red Welte upright Steinway player piano of the Denis Condon Collection of Reproducing Pianos and Rolls. \autocite{redwelte} Although this keyboard is the main medium through which one can hear composers' original performances, it represents much more than just a listening tool. Player pianos like this Welte represent three things: the first era of the mechanical automation of music, the redefinition of what ``authentic'' music creation is, and the pushing of new boundaries for what aspects of music can be preserved. \autocite{playerpiano}

Alhough the advent of technologies like robots, artificial intelligence, and automation seems to be a new and ever-increasing threat to replacing humanity, 
%increase the likelihood of robots taking our jobs and our livelihood,
\autocite{robots} the debut of the player piano in concert in Donaueschingen, Germany, on July 25, 1926, was an early harbinger of the automation of human art, threatening to replace human performers. \autocite{playerpiano} ``A contemporary account captured the strange scene as the music began:"

\begin{quote}
The hall was illuminated by unseen sources. It was absolutely quiet as Hindemith wound up the device. [ . . . ] The piano began to play: music like an \'etude, toccatas with otherwise unplayable harmonic progressions, with a speed that could never be approached even by the most virtuosic of players, with an exactitude of which a human could never be capable, with a superhuman sonic force, with a geometrical clarity of rhythm, tempo, dynamics, and phrasing, which only a machine can produce. [ . . . ] The piano finished the composition and there was an uneasy pause. Should one applaud? There's no one sitting there. It's only a machine. Finally a quiet applause, growing louder. Calls of ``da capo.'' And sure enough, the piano played it again, without hesitation, as precisely as the first time.\autocite{playerpiano}
\end{quote}

Player pianos marked the first time the ``piano player'' was taken out of ``piano playing'' and replaced with a ``player piano.'' \autocite{playerpiano} Indeed, two years before this concert, a young composer and writer named Hans Heinz Stuckenschmidt wrote a short article called ``Mechanization of Music'' concerning the authenticity of human performance.\autocite{playerpiano}

In Stuckenschmidt's view, the human performances that listeners most often affectionately refer to as ``nuanced'' and ``unpredictable'' could more accurately be characterized as performances with nothing but \textit{defects}. ``The performer's character, his momentary feelings, his private opinions are quite irrelevant to the essence of the artwork,'' he wrote. ``The more `objective' the interpreter, the better the interpretation."\autocite{playerpiano} In fact, in his mind, the ideal musician is not an interpreter at all, but rather an ``administrator'' of the composer's directions. \autocite{playerpiano}

Player pianos allowed this very philosophy to propagate, and laid the foundations for materialism in the music industry today. Elise Fellows White sums up music materialism perfectly in her piece \textit{Music versus Materialism}:

\begin{quote}
 [A]s books have become common with the invention of
 printing, so has music with the invention of the player-machines;
 and the best books and the best music must now share the same
 careless fate. Always will there be someone to value them, yet
 the dust gathers thick to-day on Dante, and Milton; while the
 young folks laugh and chatter through heavenly records of Galli
Curci or of Heifetz, without so much as a pretense at listening.
 To them it is but a diversion, associated with social hours and the
 amusements of idleness. In choosing his life-work the boy turns
 his serious attention to electricity or chemistry; the girl to problems of domestic science, nursing, or the like\ldots
Our coming generation has planted its feet all too firmly
 upon the solid earth. It has learned to fly physically, but not
 spiritually.
 My own personal belief is that the era of great musical
 invention is past. That the tunes have all been played or sung.
 That the accumulated material of past years is better worth while
 than all our feeble attempts to create a music of the future. \autocite{materialism}
\end{quote}

% TODO: create a better flow/intro to this next paragraph?

Despite this foreboding prophecy of music, the player piano could nevertheless nearly perfectly preserve a composer's ideas and intentions. Even today, the most coveted performance of a song is that of the composer or songwriter. Why should one listen to a cover of a song---the entire basis of piano performance itself---when a performance of the original exists? Although the answer to this question was relatively straightforward when a recording of a composer's original performance did not exist, the perfect player piano recordings complicate the answer to this question.

% transition, introduction to next section
With a history and background of player pianos and player piano rolls,
%and the role they played in materialism, authenticity, and preservation
we now turn to the remaining four objects in our collection---three piano roll recordings of composers' original performances, and one piano roll performance of a highly respected, modern performer.

% section on performances themselves
A common theme among each of these piano roll recordings is the ability to observe the history of piano performance ideas---how modern performances, or even performances of the composer's contemporaries, have strayed from the composer's original, written intentions. Today, in the most prestigious international piano competitions in which thousands of dollars in prize money and career opportunities are at stake, \autocite{vancliburn} modern piano performance judging standards compel the performer to stay as faithful as possible to those very original, written intentions. \autocite{intentions} However, in listening to these original recordings, one finds that the composers themselves often did not follow their own score. This observation leads to the following question: What \textit{is} a composer's ``original intentions?'' Is it the score, the so-called ``Word of God'' from which performers today strive so painstakingly never to stray? Or should a composer's own performance of his piece weigh more heavily when evaluating ``intention?'' The answers to these questions underscore the analysis of this essay, and, like the field of psychology, \autocite{foundationpsychology} potentially undermine the very foundations of modern piano performance.
%In fact, in each of these original performances, one common trend appears---the composer does not follow his own written score.

In order to understand the largely philosophical differences between piano playing then and now, we must define some terms used in describing piano technique. When describing a series of consecutive notes written with equal value or equal time spent playing each note, the execution of how equal the time the player gives each note is called \textit{evenness.} That is, if the time spent on each consecutive note is exactly the same, the piano player is playing the notes \textit{evenly}. This evenness is a main characteristic of someone with proficient \textit{technical ability} or \textit{technical soundness}. In this context, if a pianist is \textit{lazy}, he or she is not playing to the best of his or her technical ability. Similarly, even if several notes are written to be the same length, the composer may choose to make them sound \textit{connected} or \textit{not connected}. This general difference between connectedness and not-connectedness is called \textit{articulation}. \footnote{A true presentation of this essay would include a live piano demonstration of all techniques, performed and explained by me, but as of the time of submission, time does not allow for such luxuries.}
% TODO: define those terms: evenness, technical ability, technical soundness, articulation, lazy

The first piano roll recording in our collection is the George Gershwin recording. Having recorded this song between the years 1916 and 1927, Gershwin plays his own \textit{Novelette in Fourths}.\autocite{gershwin} \textit{Novelette in Fourths} is a leisurely piece that could be played as a backdrop to a movie sequence depicting a whimsical Sunday afternoon stroll in the forest backyard of someone's countryside home.

In this example, ``evenness'' or ``technical ability'' as a standard is completely different. Gershwin is not wholly technically ``sound'' or ``articulation-discriminatory'' by today's standards, but in fact much more relaxed, or even ``lazy.'' This philosophical difference in what Gershwin emphasizes in his playing sheds a new light on how his music might and should be played today.
% how? I don't see how this follows a description of his technique
In particular, the emphasis, or rather, lack of emphasis on what he wrote on his own score in his performance gives an idea as to the whole purpose of even writing his music---\textit{relaxed} enjoyment. Modern piano performance standards demand strict evenness, perfect articulation, and flawless execution, whereas these performance ideas that are so coveted today are almost missing in George Gershwin's performance. % TODO: elaborate, give examples

Next is Alexander Scriabin's \textit{Prelude Op. 11, No. 1}. \autocite{scriabin} Once again, we must define some musical terms in order to understand the peculiarities of this performance. 
% TODO: hemiola, tempo, time, meter, tempo marking
%  The \textit{time} or \textit{meter} of a piece is the number of beats per \textit{measure}. In musical notation, a bar, or \textit{measure} is a segment of time corresponding to a specific number of beats in which each beat is represented by a particular note value and the boundaries of the bar are indicated by vertical bar lines. \textit{Hemiola} is a musical figure in which, despite the written \textit{meter}, the notes are structured in a way such that a listener might infer a different meter than that written in the music. 
In music and music theory, the beat is the basic unit of time, the pulse (regularly repeating event), of the mensural level (or beat level). The \textit{tempo} or \textit{tempo marking} is simply the speed at which the music is played, typically measured in \textit{beats per minute}.

Now that we have some additional background and tools to tackle this analysis, we will directly make some observations about this recording. 

%Scriabin's \textit{Prelude} emphasizes a growing trend of twentieth century piano composing--a departure from the traditional rules and regulations regarding time and meter. In particular, Scriabin heavily uses \textit{hemiola}, as defined in the previous paragraph, to give the piece a feeling of a \textit{lack} of meter. % TODO: 3/4 instead of 4/4 or something like that

The technology of piano rolls already allowed edits, so variables like ``live performance mistakes'' or ``jitters'' are not necessarily a factor. Scriabin could have edited his own technical errors or mistakes in interpretation.

With the common theme of ``then vs. now,'' the distinction here is how liberally Scriabin plays around with the \textit{tempo}. Performers today like Mikhail Pletnev studiously follow the \textit{tempo} marking of sixty-three to seventy-six beats per minute and the placement and time to start the \textit{accelerando} and \textit{crescendo} that Scriabin himself wrote,\autocite{mikhail} but Scriabin himself gives in to his primal instincts: Although he plays only slightly faster than his written tempo marking, Scriabin starts both his \textit{accelerando} and \textit{crescendo} much earlier than he wrote himself.\autocite{scriabinscore} It almost sounds like Scriabin was in a hurry to finish playing the song.
% TODO: maybe delve into possible philosophical reasons of the time for this? (such as definition of virtuosity)
% One possible explanation is the meaning of the word \textit{virtuosity} at the time. 

The original interpretation again sprouts this question: Because so much emphasis in piano performance is on ``the composer's original intentions,'' is it ``more correct'' to follow the written score, or should a performer look to the composer's own recording? Mikhail Pletnev answers through his playing: he prefers the former. \autocite{mikhail}

The fourth object of this collection is the piano roll recording of Debussy playing his own \textit{Children's Corner No. 1}. \autocite{debussy} Debussy's first \textit{Children's Corner} is a day in the adventurous playtime of a young child. He starts with an exposition into the outside, open world, followed by an interlude of the dreams of the child's deep sleep (more accurately, a nap), and concluding with an awakening and realization of continued exploration of the world.

%This mental image of the song plays an important role in stylistic choices during performance.
% TODO: give example, and delve into a modern interpretation

Major differences between his written score and his performance are scattered throughout Debussy's recording. On a purely technical level, by today's standards, the notes are quite uneven, and even rushed. \textit{Rushing} means that the \textit{tempo}, or the speed of the music, is constantly increasing.
Similar to Scriabin's interpretation of his own song, the performance almost sounds like Debussy was in a hurry to finish playing the song, especially towards the end.

Such a ``rushed" style of playing is not necessarily out of line with the trending styles of art at the time. Debussy was a pioneer of what we now call the ``Impressionist era of piano music.'' He had a different goal in mind than his predecessors did: the music is not about ``upstanding, perfect music,'' but its purpose is to leave the listener with an ``impression'' of the image the composer is trying to convey. This general philosophy is similar to that of Impressionist painting.

The final object in this collection is Vladimir Horowitz's performance of Rachmaninoff's famous \textit{Prelude in G minor}. \autocite{horowitz} Although Rachmaninoff's original piano roll recording of his piece was not in Stanford's piano roll collection at the time of this essay's submission, Stanford's library still had an audio CD of his performance.\autocite{rachmaninoff} The roll recording of Vladimir Horowitz---a highly respected authority on all matters Rachmaninoff---serves as a starting point for directly comparing a composer's original ``intentions'' and a modern performance.
%directly observing the evolution of a piece's performance.

In his original recording, Rachmaninoff remains calm, disciplined, and almost metronomical in many parts. \autocite{rachmaninoff} In contrast, Horowitz accelerates where there is no acceleration written, and sacrifices note accuracy for sheer power and strength. \autocite{horowitz} Both do slow down right before the lyrical interlude, but both perform the ending quite differently. Horowitz ends the final flourish with what could be described as a soft, quick whisper, whereas Rachmaninoff adds notes to the song (he is the composer, after all) with two ending "Hoorah!" notes. Although Horowitz's technique is by no means flawed, it is worth noting how incredible Rachmaninoff's technique was: In certain passages, Rachmaninoff articulates his notes so well that it sounds as if he were playing five times slower. In stark contrast, Horowitz simply sacrifices the accuracy for the sheer power and anger that the piece demands.
% TODO

So what, then, do these performances tell us about how to play these pieces? Well, nothing, really. %TODO %Despite the opinions of some judges 
The idea that every year, judges reward thousands of dollars of prize money to the most faithful (or perhaps, most robotic) performer seems inane when considering that no one---not even the composer himself---truly \textit{knows} what the composer intended when he wrote the music. Perhaps we should rethink our criteria for judging the authenticity and faithfulness of classical music. Perhaps we should throw it out completely. All that this criteria produces is an endless stream young, inhuman, robotic prodigies, faithfully reproducing what ``should'' be played, what feelings ``should'' be felt, and what the composer ``truly intended,'' according to the whims of a small selection of indignant elitists (of which I am unfortunately a likely member).

%In this essay, we saw the origins of the materialism that engulfs the world of audio today and the beginnings of alternatives to ``authentic" music creation, 
In this essay, we saw how the emergence of the player piano poised the question of what an authentic piano performance really is. We also compared composers' original performances to today's standards, and even compared it with one performance of the modern day. Although the question of whether to weigh a composer's written score or his own performance more heavily in playing a piano piece remains an open question, the piano roll recordings still greatly inform and give the modern performer deep insight into the mind of the composer.

%\footnote{Although not in Stanford's collection, Rachmaninoff's original recording is available for listening here: \url{https://www.youtube.com/watch?v=M8RyWFA7PSY}}

\end{doublespace}

\large
\printbibliography

\end{document}
