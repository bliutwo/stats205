\documentclass[]{article}
\usepackage{lmodern}
\usepackage{amssymb,amsmath}
\usepackage{ifxetex,ifluatex}
\usepackage{fixltx2e} % provides \textsubscript
\ifnum 0\ifxetex 1\fi\ifluatex 1\fi=0 % if pdftex
  \usepackage[T1]{fontenc}
  \usepackage[utf8]{inputenc}
\else % if luatex or xelatex
  \ifxetex
    \usepackage{mathspec}
  \else
    \usepackage{fontspec}
  \fi
  \defaultfontfeatures{Ligatures=TeX,Scale=MatchLowercase}
\fi
% use upquote if available, for straight quotes in verbatim environments
\IfFileExists{upquote.sty}{\usepackage{upquote}}{}
% use microtype if available
\IfFileExists{microtype.sty}{%
\usepackage{microtype}
\UseMicrotypeSet[protrusion]{basicmath} % disable protrusion for tt fonts
}{}
\usepackage[margin=1in]{geometry}
\usepackage{hyperref}
\hypersetup{unicode=true,
            pdftitle={STATS 205: Homework Assignment 5},
            pdfauthor={Brian Liu},
            pdfborder={0 0 0},
            breaklinks=true}
\urlstyle{same}  % don't use monospace font for urls
\usepackage{color}
\usepackage{fancyvrb}
\newcommand{\VerbBar}{|}
\newcommand{\VERB}{\Verb[commandchars=\\\{\}]}
\DefineVerbatimEnvironment{Highlighting}{Verbatim}{commandchars=\\\{\}}
% Add ',fontsize=\small' for more characters per line
\usepackage{framed}
\definecolor{shadecolor}{RGB}{248,248,248}
\newenvironment{Shaded}{\begin{snugshade}}{\end{snugshade}}
\newcommand{\AlertTok}[1]{\textcolor[rgb]{0.94,0.16,0.16}{#1}}
\newcommand{\AnnotationTok}[1]{\textcolor[rgb]{0.56,0.35,0.01}{\textbf{\textit{#1}}}}
\newcommand{\AttributeTok}[1]{\textcolor[rgb]{0.77,0.63,0.00}{#1}}
\newcommand{\BaseNTok}[1]{\textcolor[rgb]{0.00,0.00,0.81}{#1}}
\newcommand{\BuiltInTok}[1]{#1}
\newcommand{\CharTok}[1]{\textcolor[rgb]{0.31,0.60,0.02}{#1}}
\newcommand{\CommentTok}[1]{\textcolor[rgb]{0.56,0.35,0.01}{\textit{#1}}}
\newcommand{\CommentVarTok}[1]{\textcolor[rgb]{0.56,0.35,0.01}{\textbf{\textit{#1}}}}
\newcommand{\ConstantTok}[1]{\textcolor[rgb]{0.00,0.00,0.00}{#1}}
\newcommand{\ControlFlowTok}[1]{\textcolor[rgb]{0.13,0.29,0.53}{\textbf{#1}}}
\newcommand{\DataTypeTok}[1]{\textcolor[rgb]{0.13,0.29,0.53}{#1}}
\newcommand{\DecValTok}[1]{\textcolor[rgb]{0.00,0.00,0.81}{#1}}
\newcommand{\DocumentationTok}[1]{\textcolor[rgb]{0.56,0.35,0.01}{\textbf{\textit{#1}}}}
\newcommand{\ErrorTok}[1]{\textcolor[rgb]{0.64,0.00,0.00}{\textbf{#1}}}
\newcommand{\ExtensionTok}[1]{#1}
\newcommand{\FloatTok}[1]{\textcolor[rgb]{0.00,0.00,0.81}{#1}}
\newcommand{\FunctionTok}[1]{\textcolor[rgb]{0.00,0.00,0.00}{#1}}
\newcommand{\ImportTok}[1]{#1}
\newcommand{\InformationTok}[1]{\textcolor[rgb]{0.56,0.35,0.01}{\textbf{\textit{#1}}}}
\newcommand{\KeywordTok}[1]{\textcolor[rgb]{0.13,0.29,0.53}{\textbf{#1}}}
\newcommand{\NormalTok}[1]{#1}
\newcommand{\OperatorTok}[1]{\textcolor[rgb]{0.81,0.36,0.00}{\textbf{#1}}}
\newcommand{\OtherTok}[1]{\textcolor[rgb]{0.56,0.35,0.01}{#1}}
\newcommand{\PreprocessorTok}[1]{\textcolor[rgb]{0.56,0.35,0.01}{\textit{#1}}}
\newcommand{\RegionMarkerTok}[1]{#1}
\newcommand{\SpecialCharTok}[1]{\textcolor[rgb]{0.00,0.00,0.00}{#1}}
\newcommand{\SpecialStringTok}[1]{\textcolor[rgb]{0.31,0.60,0.02}{#1}}
\newcommand{\StringTok}[1]{\textcolor[rgb]{0.31,0.60,0.02}{#1}}
\newcommand{\VariableTok}[1]{\textcolor[rgb]{0.00,0.00,0.00}{#1}}
\newcommand{\VerbatimStringTok}[1]{\textcolor[rgb]{0.31,0.60,0.02}{#1}}
\newcommand{\WarningTok}[1]{\textcolor[rgb]{0.56,0.35,0.01}{\textbf{\textit{#1}}}}
\usepackage{graphicx,grffile}
\makeatletter
\def\maxwidth{\ifdim\Gin@nat@width>\linewidth\linewidth\else\Gin@nat@width\fi}
\def\maxheight{\ifdim\Gin@nat@height>\textheight\textheight\else\Gin@nat@height\fi}
\makeatother
% Scale images if necessary, so that they will not overflow the page
% margins by default, and it is still possible to overwrite the defaults
% using explicit options in \includegraphics[width, height, ...]{}
\setkeys{Gin}{width=\maxwidth,height=\maxheight,keepaspectratio}
\IfFileExists{parskip.sty}{%
\usepackage{parskip}
}{% else
\setlength{\parindent}{0pt}
\setlength{\parskip}{6pt plus 2pt minus 1pt}
}
\setlength{\emergencystretch}{3em}  % prevent overfull lines
\providecommand{\tightlist}{%
  \setlength{\itemsep}{0pt}\setlength{\parskip}{0pt}}
\setcounter{secnumdepth}{0}
% Redefines (sub)paragraphs to behave more like sections
\ifx\paragraph\undefined\else
\let\oldparagraph\paragraph
\renewcommand{\paragraph}[1]{\oldparagraph{#1}\mbox{}}
\fi
\ifx\subparagraph\undefined\else
\let\oldsubparagraph\subparagraph
\renewcommand{\subparagraph}[1]{\oldsubparagraph{#1}\mbox{}}
\fi

%%% Use protect on footnotes to avoid problems with footnotes in titles
\let\rmarkdownfootnote\footnote%
\def\footnote{\protect\rmarkdownfootnote}

%%% Change title format to be more compact
\usepackage{titling}

% Create subtitle command for use in maketitle
\providecommand{\subtitle}[1]{
  \posttitle{
    \begin{center}\large#1\end{center}
    }
}

\setlength{\droptitle}{-2em}

  \title{STATS 205: Homework Assignment 5}
    \pretitle{\vspace{\droptitle}\centering\huge}
  \posttitle{\par}
    \author{Brian Liu}
    \preauthor{\centering\large\emph}
  \postauthor{\par}
      \predate{\centering\large\emph}
  \postdate{\par}
    \date{6/10/2019}


\begin{document}
\maketitle

\hypertarget{solution-to-problem-1}{%
\subsection{Solution to Problem 1}\label{solution-to-problem-1}}

\begin{quote}
We say that two observations \(X_1\) and \(X_2\) are \emph{independent}
of one another with respect to a collection of events \(\mathcal{A}\) if

\[
    Pr\left\{X_1 \in A \textrm{  and  } X_2 \in B\right\} = Pr\left\{X_1 \in A \right\}Pr\left\{X_2 \in B\right\}
\] where A and B are any two not necessarily distinct sets of outcomes
belonging to \(\mathcal{A}^3\).
\end{quote}

--
\href{http://www.r-tutor.com/elementary-statistics/non-parametric-methods/mann-whitney-wilcoxon-test}{2.2.1
Independent Observations; Permutation, Parametric, and Bootstrap Tests
of Hypotheses; Good, Phillip I}

\begin{quote}
In deciding whether your own observations are exchangeable and a
permutation test applicable, the key question is the one we posed in the
very first chapter: Under the null hypothesis of no differences among
the various experimental or survey groups, can we exchange the labels on
the observations without significantly affecting the results?
\end{quote}

--
\href{http://www.r-tutor.com/elementary-statistics/non-parametric-methods/mann-whitney-wilcoxon-test}{2.2.2
Exchangeable Observations; Permutation, Parametric, and Bootstrap Tests
of Hypotheses; Good, Phillip I}

\hypertarget{solution-to-problem-2}{%
\subsection{Solution to Problem 2}\label{solution-to-problem-2}}

\begin{Shaded}
\begin{Highlighting}[]
\NormalTok{cysticerci <-}\StringTok{ }\KeywordTok{c}\NormalTok{(}\FloatTok{28.9}\NormalTok{, }\FloatTok{32.8}\NormalTok{, }\FloatTok{12.0}\NormalTok{, }\FloatTok{9.9}\NormalTok{, }\FloatTok{15.0}\NormalTok{, }\FloatTok{38.0}\NormalTok{, }\FloatTok{12.5}\NormalTok{, }\FloatTok{36.5}\NormalTok{, }\FloatTok{8.6}\NormalTok{, }\FloatTok{26.8}\NormalTok{);cysticerci}
\end{Highlighting}
\end{Shaded}

\begin{verbatim}
##  [1] 28.9 32.8 12.0  9.9 15.0 38.0 12.5 36.5  8.6 26.8
\end{verbatim}

\begin{Shaded}
\begin{Highlighting}[]
\NormalTok{worms_reco <-}\StringTok{ }\KeywordTok{c}\NormalTok{(}\FloatTok{1.0}\NormalTok{, }\FloatTok{7.7}\NormalTok{, }\FloatTok{7.3}\NormalTok{, }\FloatTok{7.9}\NormalTok{, }\FloatTok{1.1}\NormalTok{, }\FloatTok{3.5}\NormalTok{, }\FloatTok{18.9}\NormalTok{, }\FloatTok{33.9}\NormalTok{, }\FloatTok{28.6}\NormalTok{, }\FloatTok{25.0}\NormalTok{); worms_reco}
\end{Highlighting}
\end{Shaded}

\begin{verbatim}
##  [1]  1.0  7.7  7.3  7.9  1.1  3.5 18.9 33.9 28.6 25.0
\end{verbatim}

\begin{Shaded}
\begin{Highlighting}[]
\KeywordTok{length}\NormalTok{(cysticerci)}
\end{Highlighting}
\end{Shaded}

\begin{verbatim}
## [1] 10
\end{verbatim}

\begin{Shaded}
\begin{Highlighting}[]
\KeywordTok{length}\NormalTok{(worms_reco)}
\end{Highlighting}
\end{Shaded}

\begin{verbatim}
## [1] 10
\end{verbatim}

The null hypothesis is that the mean weight of introduced cysticerci
\emph{has no correlation with} the mean weight of worms recovered. That
is,

\[
    H_0: \uptau = 0
\]

The alternative hypothesis is that allergic smokers have higher sputum
histamine levels than nonallergic smokers. That is,

\[
    H_0: p_a > p_n
\]

To test the null hypothesis against the alternative hypothesis, we will
use the Mann-Whitney-Wilcoxin test, since the two samples are
independent.

\begin{quote}
Two data samples are independent if they come from distinct populations
and the samples do not affect each other.
\end{quote}

--
\href{http://www.r-tutor.com/elementary-statistics/non-parametric-methods/mann-whitney-wilcoxon-test}{Mann-Whitney-Wilcoxon
Test}

The \(p\)-value is \(0.000386\), which is significant at the
\(\alpha = 0.05\) level. There is strong evidence that allergic smokers
have higher sputum histamine levels than nonallergic smokers.


\end{document}
