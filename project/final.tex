\documentclass[]{article}
\usepackage{lmodern}
\usepackage{amssymb,amsmath}
\usepackage{ifxetex,ifluatex}
\usepackage{fixltx2e} % provides \textsubscript
\ifnum 0\ifxetex 1\fi\ifluatex 1\fi=0 % if pdftex
  \usepackage[T1]{fontenc}
  \usepackage[utf8]{inputenc}
\else % if luatex or xelatex
  \ifxetex
    \usepackage{mathspec}
  \else
    \usepackage{fontspec}
  \fi
  \defaultfontfeatures{Ligatures=TeX,Scale=MatchLowercase}
\fi
% use upquote if available, for straight quotes in verbatim environments
\IfFileExists{upquote.sty}{\usepackage{upquote}}{}
% use microtype if available
\IfFileExists{microtype.sty}{%
\usepackage{microtype}
\UseMicrotypeSet[protrusion]{basicmath} % disable protrusion for tt fonts
}{}
\usepackage[margin=1in]{geometry}
\usepackage{hyperref}
\hypersetup{unicode=true,
            pdftitle={STATS 205: Final Project Write-Up},
            pdfauthor={Brian Liu},
            pdfborder={0 0 0},
            breaklinks=true}
\urlstyle{same}  % don't use monospace font for urls
\usepackage{color}
\usepackage{fancyvrb}
\newcommand{\VerbBar}{|}
\newcommand{\VERB}{\Verb[commandchars=\\\{\}]}
\DefineVerbatimEnvironment{Highlighting}{Verbatim}{commandchars=\\\{\}}
% Add ',fontsize=\small' for more characters per line
\usepackage{framed}
\definecolor{shadecolor}{RGB}{248,248,248}
\newenvironment{Shaded}{\begin{snugshade}}{\end{snugshade}}
\newcommand{\AlertTok}[1]{\textcolor[rgb]{0.94,0.16,0.16}{#1}}
\newcommand{\AnnotationTok}[1]{\textcolor[rgb]{0.56,0.35,0.01}{\textbf{\textit{#1}}}}
\newcommand{\AttributeTok}[1]{\textcolor[rgb]{0.77,0.63,0.00}{#1}}
\newcommand{\BaseNTok}[1]{\textcolor[rgb]{0.00,0.00,0.81}{#1}}
\newcommand{\BuiltInTok}[1]{#1}
\newcommand{\CharTok}[1]{\textcolor[rgb]{0.31,0.60,0.02}{#1}}
\newcommand{\CommentTok}[1]{\textcolor[rgb]{0.56,0.35,0.01}{\textit{#1}}}
\newcommand{\CommentVarTok}[1]{\textcolor[rgb]{0.56,0.35,0.01}{\textbf{\textit{#1}}}}
\newcommand{\ConstantTok}[1]{\textcolor[rgb]{0.00,0.00,0.00}{#1}}
\newcommand{\ControlFlowTok}[1]{\textcolor[rgb]{0.13,0.29,0.53}{\textbf{#1}}}
\newcommand{\DataTypeTok}[1]{\textcolor[rgb]{0.13,0.29,0.53}{#1}}
\newcommand{\DecValTok}[1]{\textcolor[rgb]{0.00,0.00,0.81}{#1}}
\newcommand{\DocumentationTok}[1]{\textcolor[rgb]{0.56,0.35,0.01}{\textbf{\textit{#1}}}}
\newcommand{\ErrorTok}[1]{\textcolor[rgb]{0.64,0.00,0.00}{\textbf{#1}}}
\newcommand{\ExtensionTok}[1]{#1}
\newcommand{\FloatTok}[1]{\textcolor[rgb]{0.00,0.00,0.81}{#1}}
\newcommand{\FunctionTok}[1]{\textcolor[rgb]{0.00,0.00,0.00}{#1}}
\newcommand{\ImportTok}[1]{#1}
\newcommand{\InformationTok}[1]{\textcolor[rgb]{0.56,0.35,0.01}{\textbf{\textit{#1}}}}
\newcommand{\KeywordTok}[1]{\textcolor[rgb]{0.13,0.29,0.53}{\textbf{#1}}}
\newcommand{\NormalTok}[1]{#1}
\newcommand{\OperatorTok}[1]{\textcolor[rgb]{0.81,0.36,0.00}{\textbf{#1}}}
\newcommand{\OtherTok}[1]{\textcolor[rgb]{0.56,0.35,0.01}{#1}}
\newcommand{\PreprocessorTok}[1]{\textcolor[rgb]{0.56,0.35,0.01}{\textit{#1}}}
\newcommand{\RegionMarkerTok}[1]{#1}
\newcommand{\SpecialCharTok}[1]{\textcolor[rgb]{0.00,0.00,0.00}{#1}}
\newcommand{\SpecialStringTok}[1]{\textcolor[rgb]{0.31,0.60,0.02}{#1}}
\newcommand{\StringTok}[1]{\textcolor[rgb]{0.31,0.60,0.02}{#1}}
\newcommand{\VariableTok}[1]{\textcolor[rgb]{0.00,0.00,0.00}{#1}}
\newcommand{\VerbatimStringTok}[1]{\textcolor[rgb]{0.31,0.60,0.02}{#1}}
\newcommand{\WarningTok}[1]{\textcolor[rgb]{0.56,0.35,0.01}{\textbf{\textit{#1}}}}
\usepackage{graphicx,grffile}
\makeatletter
\def\maxwidth{\ifdim\Gin@nat@width>\linewidth\linewidth\else\Gin@nat@width\fi}
\def\maxheight{\ifdim\Gin@nat@height>\textheight\textheight\else\Gin@nat@height\fi}
\makeatother
% Scale images if necessary, so that they will not overflow the page
% margins by default, and it is still possible to overwrite the defaults
% using explicit options in \includegraphics[width, height, ...]{}
\setkeys{Gin}{width=\maxwidth,height=\maxheight,keepaspectratio}
\IfFileExists{parskip.sty}{%
\usepackage{parskip}
}{% else
\setlength{\parindent}{0pt}
\setlength{\parskip}{6pt plus 2pt minus 1pt}
}
\setlength{\emergencystretch}{3em}  % prevent overfull lines
\providecommand{\tightlist}{%
  \setlength{\itemsep}{0pt}\setlength{\parskip}{0pt}}
\setcounter{secnumdepth}{0}
% Redefines (sub)paragraphs to behave more like sections
\ifx\paragraph\undefined\else
\let\oldparagraph\paragraph
\renewcommand{\paragraph}[1]{\oldparagraph{#1}\mbox{}}
\fi
\ifx\subparagraph\undefined\else
\let\oldsubparagraph\subparagraph
\renewcommand{\subparagraph}[1]{\oldsubparagraph{#1}\mbox{}}
\fi

%%% Use protect on footnotes to avoid problems with footnotes in titles
\let\rmarkdownfootnote\footnote%
\def\footnote{\protect\rmarkdownfootnote}

%%% Change title format to be more compact
\usepackage{titling}

% Create subtitle command for use in maketitle
\providecommand{\subtitle}[1]{
  \posttitle{
    \begin{center}\large#1\end{center}
    }
}

\setlength{\droptitle}{-2em}

  \title{STATS 205: Final Project Write-Up}
    \pretitle{\vspace{\droptitle}\centering\huge}
  \posttitle{\par}
    \author{Brian Liu}
    \preauthor{\centering\large\emph}
  \postauthor{\par}
      \predate{\centering\large\emph}
  \postdate{\par}
    \date{6/14/2019}


\begin{document}
\maketitle

\hypertarget{background-of-the-data-and-why-it-is-interesting-or-important}{%
\section{1. Background of the data and why it is interesting or
important}\label{background-of-the-data-and-why-it-is-interesting-or-important}}

\begin{quote}
The data we are using is the data from WHO suicide statistics from
Kaggle. This gives population-based statistics on suicide rate.
\end{quote}

The reason this data is interesting and important is that suicide is
prevalent in many times and places around the world, but many places and
times have different suicide rates. When it comes to suicide, there are
many potential factors or attributes that may be correlated with an
increased risk of suicide, such as:

\begin{itemize}
\tightlist
\item
  a person's sex
\item
  the age group a person belongs to
\item
  the generation a person was born in
\end{itemize}

The goal is to find significant correlations between these factors and
suicide rates: that is, does \(x\) factor positively predict suicide
rate?

The simple inspiration is suicide prevention: If we can identify the
factors that correlate positively with, or predict high suicide rates,
then we can target our suicide prevention efforts towards populations
with those high-risk factors or attributes.

\hypertarget{explanation-of-the-method-studied-and-its-properties}{%
\section{2. Explanation of the method studied and its
properties}\label{explanation-of-the-method-studied-and-its-properties}}

We will use the statistical techniques of nonparametric bootstrap and
parametric bootstrap methods to aid in prediction, with linear
regression as well, and use cross-validation to test if, given new data
for a population, this population is at risk of suicide. In other words,
predict if the suicide rate would be abnormally or significantly high,
and then compare the performance between the two methods (nonparametric
and parametric).

\hypertarget{bootstrapping}{%
\subsection{Bootstrapping}\label{bootstrapping}}

\begin{quote}
In statistics, bootstrapping is any test or metric that relies on random
sampling with replacement. Bootstrapping allows assigning measures of
accuracy (defined in terms of bias, variance, confidence intervals,
prediction error or some other such measure) to sample estimates. This
technique allows estimation of the sampling distribution of almost any
statistic using random sampling methods. Generally, it falls in the
broader class of resampling methods.

Bootstrapping is the practice of estimating properties of an estimator
(such as its variance) by measuring those properties when sampling from
an approximating distribution. One standard choice for an approximating
distribution is the empirical distribution function of the observed
data. In the case where a set of observations can be assumed to be from
an independent and identically distributed population, this can be
implemented by constructing a number of resamples with replacement, of
the observed dataset (and of equal size to the observed dataset).

It may also be used for constructing hypothesis tests. It is often used
as an alternative to statistical inference based on the assumption of a
parametric model when that assumption is in doubt, or where parametric
inference is impossible or requires complicated formulas for the
calculation of standard errors.
\end{quote}

\hypertarget{nonparametric-vs.-parametric-bootstrap}{%
\subsection{Nonparametric vs.~Parametric
bootstrap}\label{nonparametric-vs.-parametric-bootstrap}}

\hypertarget{linear-regression---kendall-rank-correlation-coefficient}{%
\subsection{Linear regression - Kendall rank correlation
coefficient}\label{linear-regression---kendall-rank-correlation-coefficient}}

\begin{quote}
In statistics, the Kendall rank correlation coefficient, commonly
referred to as Kendall's tau coefficient (after the Greek letter τ), is
a statistic used to measure the ordinal association between two measured
quantities. A tau test is a non-parametric hypothesis test for
statistical dependence based on the tau coefficient.

It is a measure of rank correlation: the similarity of the orderings of
the data when ranked by each of the quantities. It is named after
Maurice Kendall, who developed it in 1938,{[}1{]} though Gustav Fechner
had proposed a similar measure in the context of time series in
1897.{[}2{]}

Intuitively, the Kendall correlation between two variables will be high
when observations have a similar (or identical for a correlation of 1)
rank (i.e.~relative position label of the observations within the
variable: 1st, 2nd, 3rd, etc.) between the two variables, and low when
observations have a dissimilar (or fully different for a correlation of
-1) rank between the two variables.

Both Kendall's \{\displaystyle \tau \} \tau  and Spearman's
\{\displaystyle \rho \} \rho  can be formulated as special cases of a
more general correlation coefficient.
\end{quote}

\hypertarget{cross-validation}{%
\subsection{Cross validation}\label{cross-validation}}

\hypertarget{data-analysis-or-simulation-study}{%
\section{3. Data analysis or simulation
study}\label{data-analysis-or-simulation-study}}

We will use the crude rate of suicide per 100,000 people.

\begin{quote}
This analysis provides information on age-standardized rates\ldots{}
\end{quote}

\begin{Shaded}
\begin{Highlighting}[]
\NormalTok{who_suicide_statistics_df <-}\StringTok{ }\KeywordTok{read.csv}\NormalTok{(}\StringTok{"who_suicide_statistics.csv"}\NormalTok{)}
\KeywordTok{head}\NormalTok{(who_suicide_statistics_df)}
\end{Highlighting}
\end{Shaded}

\begin{verbatim}
##   country year    sex         age suicides_no population
## 1 Albania 1985 female 15-24 years          NA     277900
## 2 Albania 1985 female 25-34 years          NA     246800
## 3 Albania 1985 female 35-54 years          NA     267500
## 4 Albania 1985 female  5-14 years          NA     298300
## 5 Albania 1985 female 55-74 years          NA     138700
## 6 Albania 1985 female   75+ years          NA      34200
\end{verbatim}

\begin{Shaded}
\begin{Highlighting}[]
\KeywordTok{colnames}\NormalTok{(who_suicide_statistics_df)}
\end{Highlighting}
\end{Shaded}

\begin{verbatim}
## [1] "country"     "year"        "sex"         "age"         "suicides_no"
## [6] "population"
\end{verbatim}

\begin{quote}
Filter and save countries with missing suicide rate.
\end{quote}

\begin{Shaded}
\begin{Highlighting}[]
\KeywordTok{library}\NormalTok{(tidyverse)}
\end{Highlighting}
\end{Shaded}

\begin{verbatim}
## Registered S3 methods overwritten by 'ggplot2':
##   method         from 
##   [.quosures     rlang
##   c.quosures     rlang
##   print.quosures rlang
\end{verbatim}

\begin{verbatim}
## -- Attaching packages ---------------------------------- tidyverse 1.2.1 --
\end{verbatim}

\begin{verbatim}
## v ggplot2 3.1.1     v purrr   0.3.2
## v tibble  2.1.1     v dplyr   0.8.1
## v tidyr   0.8.3     v stringr 1.4.0
## v readr   1.3.1     v forcats 0.4.0
\end{verbatim}

\begin{verbatim}
## -- Conflicts ------------------------------------- tidyverse_conflicts() --
## x dplyr::filter() masks stats::filter()
## x dplyr::lag()    masks stats::lag()
\end{verbatim}

\begin{Shaded}
\begin{Highlighting}[]
\NormalTok{filtered_suicide_df <-}\StringTok{ }\KeywordTok{drop_na}\NormalTok{(who_suicide_statistics_df, }\StringTok{"suicides_no"}\NormalTok{)}
\KeywordTok{head}\NormalTok{(filtered_suicide_df)}
\end{Highlighting}
\end{Shaded}

\begin{verbatim}
##    country year    sex         age suicides_no population
## 25 Albania 1987 female 15-24 years          14     289700
## 26 Albania 1987 female 25-34 years           4     257200
## 27 Albania 1987 female 35-54 years           6     278800
## 28 Albania 1987 female  5-14 years           0     311000
## 29 Albania 1987 female 55-74 years           0     144600
## 30 Albania 1987 female   75+ years           1      35600
\end{verbatim}

\begin{quote}
After filtering countries with missing suicide rate, take a random
sample of 100 countries and make sure each continent has approximately
equal countries.
\end{quote}

Filter countries by continent:

\begin{Shaded}
\begin{Highlighting}[]
\KeywordTok{library}\NormalTok{(countrycode)}
\NormalTok{filtered_suicide_df}\OperatorTok{$}\NormalTok{continent <-}\StringTok{ }\KeywordTok{countrycode}\NormalTok{(}\DataTypeTok{sourcevar =}\NormalTok{ filtered_suicide_df[, }\StringTok{"country"}\NormalTok{],}
                            \DataTypeTok{origin =} \StringTok{"country.name"}\NormalTok{,}
                            \DataTypeTok{destination =} \StringTok{"continent"}\NormalTok{)}
\end{Highlighting}
\end{Shaded}

\begin{verbatim}
## Warning in countrycode(sourcevar = filtered_suicide_df[, "country"], origin = "country.name", : Some values were not matched unambiguously: Rodrigues, Virgin Islands (USA)
\end{verbatim}

\begin{verbatim}
## Warning in countrycode(sourcevar = filtered_suicide_df[, "country"], origin = "country.name", : Some strings were matched more than once, and therefore set to <NA> in the result: Virgin Islands (USA),Americas,Americas
\end{verbatim}

\begin{Shaded}
\begin{Highlighting}[]
\KeywordTok{head}\NormalTok{(filtered_suicide_df)}
\end{Highlighting}
\end{Shaded}

\begin{verbatim}
##    country year    sex         age suicides_no population continent
## 25 Albania 1987 female 15-24 years          14     289700    Europe
## 26 Albania 1987 female 25-34 years           4     257200    Europe
## 27 Albania 1987 female 35-54 years           6     278800    Europe
## 28 Albania 1987 female  5-14 years           0     311000    Europe
## 29 Albania 1987 female 55-74 years           0     144600    Europe
## 30 Albania 1987 female   75+ years           1      35600    Europe
\end{verbatim}

\begin{Shaded}
\begin{Highlighting}[]
\KeywordTok{write.csv}\NormalTok{(filtered_suicide_df, }\StringTok{'filtered_suicide.csv'}\NormalTok{)}
\end{Highlighting}
\end{Shaded}

Let us find out which continents are counted:

\begin{Shaded}
\begin{Highlighting}[]
\CommentTok{# Get list of continents}
\NormalTok{list_of_continents <-}\StringTok{ }\KeywordTok{unique}\NormalTok{(filtered_suicide_df}\OperatorTok{$}\NormalTok{continent); list_of_continents}
\end{Highlighting}
\end{Shaded}

\begin{verbatim}
## [1] "Europe"   "Americas" "Asia"     "Oceania"  "Africa"   NA
\end{verbatim}

Therefore,

\[
    \frac{100 \textrm{ countries}}{6 \textrm{ continents}} \approx 16 \textrm{ to } 17 \textrm{ countries per continent}
\]

we should randomly sample 17 countries from each continent.

Notably, there are countries that are not on any of the listed
continents. Let us see which ones those are:

\begin{Shaded}
\begin{Highlighting}[]
\NormalTok{not_in_a_continent =}\StringTok{ }\NormalTok{filtered_suicide_df[}\KeywordTok{is.na}\NormalTok{(filtered_suicide_df}\OperatorTok{$}\NormalTok{continent),]}
\KeywordTok{write.csv}\NormalTok{(not_in_a_continent, }\StringTok{'not_in_a_continent.csv'}\NormalTok{)}
\KeywordTok{head}\NormalTok{(not_in_a_continent)}
\end{Highlighting}
\end{Shaded}

\begin{verbatim}
##         country year    sex         age suicides_no population continent
## 32317 Rodrigues 2001 female 15-24 years           0         NA      <NA>
## 32318 Rodrigues 2001 female 25-34 years           0         NA      <NA>
## 32319 Rodrigues 2001 female 35-54 years           0         NA      <NA>
## 32320 Rodrigues 2001 female  5-14 years           0         NA      <NA>
## 32321 Rodrigues 2001 female 55-74 years           0         NA      <NA>
## 32322 Rodrigues 2001 female   75+ years           0         NA      <NA>
\end{verbatim}

\begin{Shaded}
\begin{Highlighting}[]
\KeywordTok{unique}\NormalTok{(not_in_a_continent}\OperatorTok{$}\NormalTok{country)}
\end{Highlighting}
\end{Shaded}

\begin{verbatim}
## [1] Rodrigues            Virgin Islands (USA)
## 141 Levels: Albania Anguilla Antigua and Barbuda Argentina ... Zimbabwe
\end{verbatim}

Let us make the choice not to include these countries in the analysis,
since there are only two countries.

\begin{Shaded}
\begin{Highlighting}[]
\CommentTok{# Take off `NA` from list of continents}
\NormalTok{list_of_continents <-}\StringTok{ }\NormalTok{list_of_continents[}\OperatorTok{-}\KeywordTok{length}\NormalTok{(list_of_continents)]}
\NormalTok{list_of_continents}
\end{Highlighting}
\end{Shaded}

\begin{verbatim}
## [1] "Europe"   "Americas" "Asia"     "Oceania"  "Africa"
\end{verbatim}

We will now create six dataframes, filtered by list of countries for
each continent.

\begin{Shaded}
\begin{Highlighting}[]
\CommentTok{# library(rlist)}
\NormalTok{countries_per_continent <-}\StringTok{ }\KeywordTok{list}\NormalTok{()}

\ControlFlowTok{for}\NormalTok{ (i }\ControlFlowTok{in} \KeywordTok{seq_along}\NormalTok{(list_of_continents))}
\NormalTok{\{}
\NormalTok{    countries_per_continent[[i]] <-}\StringTok{ }\NormalTok{filtered_suicide_df[filtered_suicide_df}\OperatorTok{$}\NormalTok{continent }\OperatorTok{==}\StringTok{ }\NormalTok{list_of_continents[i],]}
\NormalTok{\}}

\KeywordTok{length}\NormalTok{(countries_per_continent)}
\end{Highlighting}
\end{Shaded}

\begin{verbatim}
## [1] 5
\end{verbatim}

\begin{Shaded}
\begin{Highlighting}[]
\KeywordTok{length}\NormalTok{(countries_per_continent)}
\end{Highlighting}
\end{Shaded}

\begin{verbatim}
## [1] 5
\end{verbatim}

\begin{Shaded}
\begin{Highlighting}[]
\ControlFlowTok{for}\NormalTok{ (i }\ControlFlowTok{in} \KeywordTok{seq_along}\NormalTok{(countries_per_continent))}
\NormalTok{\{}
    \KeywordTok{print}\NormalTok{(}\KeywordTok{head}\NormalTok{(countries_per_continent[[i]]))}
    \KeywordTok{print}\NormalTok{(}\KeywordTok{length}\NormalTok{(countries_per_continent[[i]]))}
    \KeywordTok{cat}\NormalTok{(}\StringTok{"}\CharTok{\textbackslash{}n}\StringTok{"}\NormalTok{)}
\NormalTok{\}}
\end{Highlighting}
\end{Shaded}

\begin{verbatim}
##    country year    sex         age suicides_no population continent
## 25 Albania 1987 female 15-24 years          14     289700    Europe
## 26 Albania 1987 female 25-34 years           4     257200    Europe
## 27 Albania 1987 female 35-54 years           6     278800    Europe
## 28 Albania 1987 female  5-14 years           0     311000    Europe
## 29 Albania 1987 female 55-74 years           0     144600    Europe
## 30 Albania 1987 female   75+ years           1      35600    Europe
## [1] 7
## 
##      country year    sex         age suicides_no population continent
## 373 Anguilla 1983 female 15-24 years           0         NA  Americas
## 374 Anguilla 1983 female 25-34 years           0         NA  Americas
## 375 Anguilla 1983 female 35-54 years           0         NA  Americas
## 376 Anguilla 1983 female  5-14 years           0         NA  Americas
## 377 Anguilla 1983 female 55-74 years           0         NA  Americas
## 378 Anguilla 1983 female   75+ years           0         NA  Americas
## [1] 7
## 
##      country year    sex         age suicides_no population continent
## 1501 Armenia 1981 female 15-24 years           5     348000      Asia
## 1502 Armenia 1981 female 25-34 years           6     242200      Asia
## 1503 Armenia 1981 female 35-54 years           6     333500      Asia
## 1504 Armenia 1981 female  5-14 years           0     295200      Asia
## 1505 Armenia 1981 female 55-74 years          10     164300      Asia
## 1506 Armenia 1981 female   75+ years           7      43100      Asia
## [1] 7
## 
##        country year    sex         age suicides_no population continent
## 2161 Australia 1979 female 15-24 years          71    1236800   Oceania
## 2162 Australia 1979 female 25-34 years          86    1138500   Oceania
## 2163 Australia 1979 female 35-54 years         171    1572100   Oceania
## 2164 Australia 1979 female  5-14 years           1    1246500   Oceania
## 2165 Australia 1979 female 55-74 years         135    1137800   Oceania
## 2166 Australia 1979 female   75+ years          15     309900   Oceania
## [1] 7
## 
##         country year    sex         age suicides_no population continent
## 7669 Cabo Verde 2011 female 15-24 years           1      56039    Africa
## 7670 Cabo Verde 2011 female 25-34 years           0      38528    Africa
## 7671 Cabo Verde 2011 female 35-54 years           2      49078    Africa
## 7672 Cabo Verde 2011 female  5-14 years           0      56558    Africa
## 7673 Cabo Verde 2011 female 55-74 years           2      19887    Africa
## 7674 Cabo Verde 2011 female   75+ years           0       7582    Africa
## [1] 7
\end{verbatim}

\href{https://stackoverflow.com/questions/14987743/head-function-does-not-work-within-for-loop}{This
text links to very important information about why a \texttt{for} loop
doesn't print anything.}\footnote{Basically, \texttt{for} loops are
  functions themselves. R prints out the result of a command
  automatically, but functions are not inherently a command, and since
  \texttt{for} loops are functions, nothing will be printed. The
  solution is to have \texttt{print(command())} within the \texttt{for}
  loop to get output for your \texttt{for} loop. You will never again
  spend hours trying to find out why a \texttt{for} loop doesn't print
  anything because you're no longer an R newbie.}

\href{https://rmarkdown.rstudio.com/authoring_pandoc_markdown.html\%23raw-tex}{Link
to Pandoc Markdown formatting}

Randomly sample 17 countries from each continent:

\begin{Shaded}
\begin{Highlighting}[]
\NormalTok{list_of_continents}
\end{Highlighting}
\end{Shaded}

\begin{verbatim}
## [1] "Europe"   "Americas" "Asia"     "Oceania"  "Africa"
\end{verbatim}

\begin{Shaded}
\begin{Highlighting}[]
\ControlFlowTok{for}\NormalTok{ (i }\ControlFlowTok{in} \KeywordTok{seq_along}\NormalTok{(countries_per_continent))}
\NormalTok{\{}
    \KeywordTok{print}\NormalTok{(list_of_continents[i])}
\NormalTok{    countries <-}\StringTok{ }\KeywordTok{unique}\NormalTok{(countries_per_continent[[i]]}\OperatorTok{$}\NormalTok{country)}
    \KeywordTok{print}\NormalTok{(countries)}
    \KeywordTok{print}\NormalTok{(}\KeywordTok{length}\NormalTok{(countries))}
    \KeywordTok{cat}\NormalTok{(}\StringTok{"}\CharTok{\textbackslash{}n}\StringTok{"}\NormalTok{)}
\NormalTok{\}}
\end{Highlighting}
\end{Shaded}

\begin{verbatim}
## [1] "Europe"
##  [1] Albania                Austria                Belarus               
##  [4] Belgium                Bosnia and Herzegovina Bulgaria              
##  [7] Croatia                Czech Republic         Denmark               
## [10] Estonia                Finland                France                
## [13] Germany                Greece                 Hungary               
## [16] Iceland                Ireland                Italy                 
## [19] Latvia                 Lithuania              Luxembourg            
## [22] Malta                  Monaco                 Montenegro            
## [25] Netherlands            Norway                 Poland                
## [28] Portugal               Republic of Moldova    <NA>                  
## [31] Romania                Russian Federation     San Marino            
## [34] Serbia                 Slovakia               Slovenia              
## [37] Spain                  Sweden                 Switzerland           
## [40] TFYR Macedonia         Ukraine                United Kingdom        
## 141 Levels: Albania Anguilla Antigua and Barbuda Argentina ... Zimbabwe
## [1] 42
## 
## [1] "Americas"
##  [1] Anguilla                           Antigua and Barbuda               
##  [3] Argentina                          Aruba                             
##  [5] Bahamas                            Barbados                          
##  [7] Belize                             Bermuda                           
##  [9] Bolivia                            Brazil                            
## [11] British Virgin Islands             Canada                            
## [13] Cayman Islands                     Chile                             
## [15] Colombia                           Costa Rica                        
## [17] Cuba                               Dominica                          
## [19] Dominican Republic                 Ecuador                           
## [21] El Salvador                        Falkland Islands (Malvinas)       
## [23] French Guiana                      Grenada                           
## [25] Guadeloupe                         Guatemala                         
## [27] Guyana                             Haiti                             
## [29] Honduras                           Jamaica                           
## [31] Martinique                         Mexico                            
## [33] Montserrat                         Netherlands Antilles              
## [35] Nicaragua                          Panama                            
## [37] Paraguay                           Peru                              
## [39] Puerto Rico                        <NA>                              
## [41] Saint Kitts and Nevis              Saint Lucia                       
## [43] Saint Pierre and Miquelon          Saint Vincent and Grenadines      
## [45] Suriname                           Trinidad and Tobago               
## [47] Turks and Caicos Islands           United States of America          
## [49] Uruguay                            Venezuela (Bolivarian Republic of)
## 141 Levels: Albania Anguilla Antigua and Barbuda Argentina ... Zimbabwe
## [1] 50
## 
## [1] "Asia"
##  [1] Armenia                        Azerbaijan                    
##  [3] Bahrain                        Brunei Darussalam             
##  [5] Cyprus                         Georgia                       
##  [7] Hong Kong SAR                  Iran (Islamic Rep of)         
##  [9] Iraq                           Israel                        
## [11] Japan                          Jordan                        
## [13] Kazakhstan                     Kuwait                        
## [15] Kyrgyzstan                     Macau                         
## [17] Malaysia                       Maldives                      
## [19] Mongolia                       Occupied Palestinian Territory
## [21] Oman                           Philippines                   
## [23] Qatar                          Republic of Korea             
## [25] <NA>                           Saudi Arabia                  
## [27] Singapore                      Sri Lanka                     
## [29] Syrian Arab Republic           Tajikistan                    
## [31] Thailand                       Turkey                        
## [33] Turkmenistan                   United Arab Emirates          
## [35] Uzbekistan                    
## 141 Levels: Albania Anguilla Antigua and Barbuda Argentina ... Zimbabwe
## [1] 35
## 
## [1] "Oceania"
## [1] Australia   Fiji        Kiribati    New Zealand <NA>       
## 141 Levels: Albania Anguilla Antigua and Barbuda Argentina ... Zimbabwe
## [1] 5
## 
## [1] "Africa"
##  [1] Cabo Verde            Egypt                 Mauritius            
##  [4] Mayotte               Morocco               Reunion              
##  [7] <NA>                  Sao Tome and Principe Seychelles           
## [10] South Africa          Tunisia               Zimbabwe             
## 141 Levels: Albania Anguilla Antigua and Barbuda Argentina ... Zimbabwe
## [1] 12
\end{verbatim}

Since there are only 5 countries in Oceania and 12 countries in Africa,
we will use all 5 countries of Oceania and all 12 countries of Africa.

\begin{Shaded}
\begin{Highlighting}[]
\NormalTok{samples_of_countries <-}\StringTok{ }\KeywordTok{list}\NormalTok{()}
\NormalTok{num_samples <-}\StringTok{ }\DecValTok{17}
\ControlFlowTok{for}\NormalTok{ (i }\ControlFlowTok{in} \KeywordTok{seq_along}\NormalTok{(countries_per_continent))}
\NormalTok{\{}
\NormalTok{    countries <-}\StringTok{ }\KeywordTok{unique}\NormalTok{(countries_per_continent[[i]]}\OperatorTok{$}\NormalTok{country)}
\NormalTok{    current_sample <-}\StringTok{ }\KeywordTok{list}\NormalTok{()}
    \ControlFlowTok{if}\NormalTok{ (}\KeywordTok{length}\NormalTok{(countries) }\OperatorTok{>=}\StringTok{ }\NormalTok{num_samples)}
\NormalTok{    \{}
\NormalTok{        current_sample <-}\StringTok{ }\KeywordTok{sample}\NormalTok{(countries, }\DecValTok{17}\NormalTok{)}
\NormalTok{    \} }\ControlFlowTok{else}\NormalTok{ \{}
\NormalTok{        current_sample <-}\StringTok{ }\KeywordTok{sample}\NormalTok{(countries, }\KeywordTok{length}\NormalTok{(countries))}
\NormalTok{    \}}
\NormalTok{    samples_of_countries[[i]] <-}\StringTok{ }\NormalTok{current_sample}
\NormalTok{\}}
\end{Highlighting}
\end{Shaded}

Let's see the countries that we will be sampling:

\begin{Shaded}
\begin{Highlighting}[]
\NormalTok{total <-}\StringTok{ }\DecValTok{0}
\ControlFlowTok{for}\NormalTok{ (i }\ControlFlowTok{in} \KeywordTok{seq_along}\NormalTok{(samples_of_countries))}
\NormalTok{\{}
    \KeywordTok{print}\NormalTok{(list_of_continents[i])}
    \KeywordTok{print}\NormalTok{(samples_of_countries[[i]])}
    \KeywordTok{print}\NormalTok{(}\KeywordTok{length}\NormalTok{(samples_of_countries[[i]]))}
\NormalTok{    total <-}\StringTok{ }\NormalTok{total }\OperatorTok{+}\StringTok{ }\KeywordTok{length}\NormalTok{(samples_of_countries[[i]])}
    \KeywordTok{cat}\NormalTok{(}\StringTok{"}\CharTok{\textbackslash{}n}\StringTok{"}\NormalTok{)}
\NormalTok{\}}
\end{Highlighting}
\end{Shaded}

\begin{verbatim}
## [1] "Europe"
##  [1] Czech Republic San Marino     Ireland        Slovakia      
##  [5] Latvia         Sweden         Monaco         Albania       
##  [9] Montenegro     Italy          Croatia        Finland       
## [13] Belgium        Iceland        Poland         Ukraine       
## [17] Lithuania     
## 141 Levels: Albania Anguilla Antigua and Barbuda Argentina ... Zimbabwe
## [1] 17
## 
## [1] "Americas"
##  [1] Venezuela (Bolivarian Republic of) Aruba                             
##  [3] United States of America           Haiti                             
##  [5] Suriname                           Colombia                          
##  [7] Jamaica                            British Virgin Islands            
##  [9] Saint Vincent and Grenadines       Grenada                           
## [11] Belize                             Puerto Rico                       
## [13] Trinidad and Tobago                Netherlands Antilles              
## [15] Uruguay                            Bolivia                           
## [17] Brazil                            
## 141 Levels: Albania Anguilla Antigua and Barbuda Argentina ... Zimbabwe
## [1] 17
## 
## [1] "Asia"
##  [1] Tajikistan                     Occupied Palestinian Territory
##  [3] Armenia                        Japan                         
##  [5] Mongolia                       <NA>                          
##  [7] Singapore                      Kyrgyzstan                    
##  [9] Iraq                           Uzbekistan                    
## [11] Georgia                        Turkmenistan                  
## [13] Saudi Arabia                   Macau                         
## [15] Qatar                          Hong Kong SAR                 
## [17] Israel                        
## 141 Levels: Albania Anguilla Antigua and Barbuda Argentina ... Zimbabwe
## [1] 17
## 
## [1] "Oceania"
## [1] Australia   New Zealand <NA>        Kiribati    Fiji       
## 141 Levels: Albania Anguilla Antigua and Barbuda Argentina ... Zimbabwe
## [1] 5
## 
## [1] "Africa"
##  [1] South Africa          Tunisia               Sao Tome and Principe
##  [4] Cabo Verde            Seychelles            Morocco              
##  [7] <NA>                  Mayotte               Egypt                
## [10] Mauritius             Zimbabwe              Reunion              
## 141 Levels: Albania Anguilla Antigua and Barbuda Argentina ... Zimbabwe
## [1] 12
\end{verbatim}

\begin{Shaded}
\begin{Highlighting}[]
\NormalTok{total}
\end{Highlighting}
\end{Shaded}

\begin{verbatim}
## [1] 68
\end{verbatim}

Let's filter the original dataframe only to include countries that we
have sampled:

\begin{Shaded}
\begin{Highlighting}[]
\NormalTok{countries_to_test <-}\StringTok{ }\KeywordTok{list}\NormalTok{()}
\NormalTok{a <-}\StringTok{ }\DecValTok{0}
\ControlFlowTok{for}\NormalTok{ (i }\ControlFlowTok{in} \KeywordTok{seq_along}\NormalTok{(samples_of_countries))}
\NormalTok{\{}
    \CommentTok{# find out a way to access each country name}
    \CommentTok{# print each country name}
    \ControlFlowTok{for}\NormalTok{ (j }\ControlFlowTok{in} \KeywordTok{seq_along}\NormalTok{(samples_of_countries[[i]]))}
\NormalTok{    \{}
\NormalTok{        sample <-}\StringTok{ }\NormalTok{samples_of_countries[[i]]}
\NormalTok{        country_string <-}\StringTok{ }\KeywordTok{toString}\NormalTok{(sample[[j]])}
\NormalTok{        countries_to_test[a] <-}\StringTok{ }\NormalTok{country_string}
\NormalTok{        a <-}\StringTok{ }\NormalTok{a }\OperatorTok{+}\StringTok{ }\DecValTok{1}
\NormalTok{    \}}
\NormalTok{\}}

\KeywordTok{length}\NormalTok{(countries_to_test)}
\end{Highlighting}
\end{Shaded}

\begin{verbatim}
## [1] 67
\end{verbatim}

\begin{Shaded}
\begin{Highlighting}[]
\NormalTok{countries_to_test}
\end{Highlighting}
\end{Shaded}

\begin{verbatim}
## [[1]]
## [1] "San Marino"
## 
## [[2]]
## [1] "Ireland"
## 
## [[3]]
## [1] "Slovakia"
## 
## [[4]]
## [1] "Latvia"
## 
## [[5]]
## [1] "Sweden"
## 
## [[6]]
## [1] "Monaco"
## 
## [[7]]
## [1] "Albania"
## 
## [[8]]
## [1] "Montenegro"
## 
## [[9]]
## [1] "Italy"
## 
## [[10]]
## [1] "Croatia"
## 
## [[11]]
## [1] "Finland"
## 
## [[12]]
## [1] "Belgium"
## 
## [[13]]
## [1] "Iceland"
## 
## [[14]]
## [1] "Poland"
## 
## [[15]]
## [1] "Ukraine"
## 
## [[16]]
## [1] "Lithuania"
## 
## [[17]]
## [1] "Venezuela (Bolivarian Republic of)"
## 
## [[18]]
## [1] "Aruba"
## 
## [[19]]
## [1] "United States of America"
## 
## [[20]]
## [1] "Haiti"
## 
## [[21]]
## [1] "Suriname"
## 
## [[22]]
## [1] "Colombia"
## 
## [[23]]
## [1] "Jamaica"
## 
## [[24]]
## [1] "British Virgin Islands"
## 
## [[25]]
## [1] "Saint Vincent and Grenadines"
## 
## [[26]]
## [1] "Grenada"
## 
## [[27]]
## [1] "Belize"
## 
## [[28]]
## [1] "Puerto Rico"
## 
## [[29]]
## [1] "Trinidad and Tobago"
## 
## [[30]]
## [1] "Netherlands Antilles"
## 
## [[31]]
## [1] "Uruguay"
## 
## [[32]]
## [1] "Bolivia"
## 
## [[33]]
## [1] "Brazil"
## 
## [[34]]
## [1] "Tajikistan"
## 
## [[35]]
## [1] "Occupied Palestinian Territory"
## 
## [[36]]
## [1] "Armenia"
## 
## [[37]]
## [1] "Japan"
## 
## [[38]]
## [1] "Mongolia"
## 
## [[39]]
## [1] "NA"
## 
## [[40]]
## [1] "Singapore"
## 
## [[41]]
## [1] "Kyrgyzstan"
## 
## [[42]]
## [1] "Iraq"
## 
## [[43]]
## [1] "Uzbekistan"
## 
## [[44]]
## [1] "Georgia"
## 
## [[45]]
## [1] "Turkmenistan"
## 
## [[46]]
## [1] "Saudi Arabia"
## 
## [[47]]
## [1] "Macau"
## 
## [[48]]
## [1] "Qatar"
## 
## [[49]]
## [1] "Hong Kong SAR"
## 
## [[50]]
## [1] "Israel"
## 
## [[51]]
## [1] "Australia"
## 
## [[52]]
## [1] "New Zealand"
## 
## [[53]]
## [1] "NA"
## 
## [[54]]
## [1] "Kiribati"
## 
## [[55]]
## [1] "Fiji"
## 
## [[56]]
## [1] "South Africa"
## 
## [[57]]
## [1] "Tunisia"
## 
## [[58]]
## [1] "Sao Tome and Principe"
## 
## [[59]]
## [1] "Cabo Verde"
## 
## [[60]]
## [1] "Seychelles"
## 
## [[61]]
## [1] "Morocco"
## 
## [[62]]
## [1] "NA"
## 
## [[63]]
## [1] "Mayotte"
## 
## [[64]]
## [1] "Egypt"
## 
## [[65]]
## [1] "Mauritius"
## 
## [[66]]
## [1] "Zimbabwe"
## 
## [[67]]
## [1] "Reunion"
\end{verbatim}

\hypertarget{interpretation-of-the-results-or-discussion}{%
\section{4. Interpretation of the results or
discussion}\label{interpretation-of-the-results-or-discussion}}

\hypertarget{references}{%
\section{5. References}\label{references}}


\end{document}
